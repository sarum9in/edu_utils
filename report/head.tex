\documentclass[11pt]{article}
\usepackage{xltxtra}
\usepackage{polyglossia}
\setdefaultlanguage[spelling=modern]{russian}
%\setmainfont[Mapping=tex-text]{DejaVu Sans}
%\setmainfont[Mapping=tex-text]{Liberation Sans}
\setmonofont[Mapping=tex-text]{DejaVu Sans Mono}
\setmainfont[Mapping=tex-text]{Linux Libertine O}
%\setmonofont[Mapping=tex-text]{Liberation Mono}

\usepackage{etoolbox}

\usepackage{verbatim}
\usepackage{tabularx}
\usepackage{float}
\usepackage{url}

\usepackage{indentfirst}

\usepackage{algorithm}
\usepackage{algorithmic}

\usepackage[a4paper]{geometry}
\geometry{left=25mm}
\geometry{right=25mm}
\geometry{top=25mm}
\geometry{bottom=15mm}

\usepackage{listings}

\expandafter\def\expandafter\UrlBreaks\expandafter{\UrlBreaks
  \do\a\do\b\do\c\do\d\do\e\do\f\do\g\do\h\do\i\do\j
  \do\k\do\l\do\m\do\n\do\o\do\p\do\q\do\r\do\s\do\t
  \do\u\do\v\do\w\do\x\do\y\do\z\do\A\do\B\do\C\do\D
  \do\E\do\F\do\G\do\H\do\I\do\J\do\K\do\L\do\M\do\N
  \do\O\do\P\do\Q\do\R\do\S\do\T\do\U\do\V\do\W\do\X
  \do\Y\do\Z}

\setlength{\parindent}{10mm}
\makeatletter
% Переопределение команды секции
\renewcommand{\section}{\@startsection{section}{1}%
{\parindent}{3.25ex plus 1ex minus .2ex}%
{1.5ex plus .2ex}{\bfseries\large\uppercase}}

% Переопределение команды подсекции
\renewcommand{\subsection}{\@startsection{subsection}{2}%
{\parindent}{3.25ex plus 1ex minus .2ex}%
{1.5ex plus .2ex}{\bfseries}}
\makeatother

%\newcommand{\msection}[1]{\section[#1]{#1}}
%\newcommand{\msection}[1]{\section[#1]{\uppercase{#1}}}

%opening
%\title{}
%\author{Филиппов Алексей Николаевич}

\newcommand{\includepicture}[3]{
\begin{figure}[H]
\begin{center}
\leavevmode
%\large{\textbf{#2}}
\includegraphics[width=#3\textwidth]{#1}
\end{center}
\caption{#2}
\end{figure}
}

\lstset{ %
language=C++,                   % the language of the code
basicstyle=\tiny,               % the size of the fonts that are used for the code
numbers=left,                   % where to put the line-numbers
numberstyle=\footnotesize,      % the size of the fonts that are used for the line-numbers
stepnumber=2,                   % the step between two line-numbers. If it's 1, each line
                                % will be numbered
numbersep=5pt,                  % how far the line-numbers are from the code
%backgroundcolor=\color{white},  % choose the background color. You must add \usepackage{color}
showspaces=false,               % show spaces adding particular underscores
showstringspaces=false,         % underline spaces within strings
showtabs=false,                 % show tabs within strings adding particular underscores
frame=single,                   % adds a frame around the code
tabsize=2,                      % sets default tabsize to 2 spaces
captionpos=b,                   % sets the caption-position to bottom
breaklines=true,                % sets automatic line breaking
breakatwhitespace=false,        % sets if automatic breaks should only happen at whitespace
title=\lstname,                 % show the filename of files included with \lstinputlisting;
                                % also try caption instead of title
escapeinside={\%*}{*)},         % if you want to add a comment within your code
morekeywords={*,...}            % if you want to add more keywords to the set
}

